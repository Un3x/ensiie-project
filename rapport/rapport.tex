\documentclass[french]{article}
\usepackage[T1]{fontenc}
\usepackage[utf8]{inputenc}
\usepackage{lmodern}
\usepackage[french]{babel}
\selectlanguage{french}
\usepackage{amsmath}
\usepackage{float}
\usepackage{amssymb}
\usepackage{hyperref}
\usepackage{xcolor}
\usepackage{graphicx}
\usepackage[a4paper]{geometry}
\usepackage{minted}
\usepackage{listings}

\hypersetup{
	colorlinks,
	linkcolor={red!50!black},
	citecolor={blue!50!black},
	urlcolor={blue!80!black}
}
\author{Loïc DUBARD, Quentin Japhet, Aurélien Ignacio, Oscar Potier}
\title{Rapport du projet de programmation web : \\
\textbf{Un site pour gérer les points associatifs}.}

\begin{document}
\maketitle
\tableofcontents
\clearpage
\section*{Introduction}
Le but premier est de répondre au besoin du Bureau des élèves de l'ENSIIE qui consiste en un moyen de centraliser et de faciliter la constitution du classement des participation aux associations.\\

 Ce classement doit être exportable dans un fichier csv pour des manipulations externes des coefficients des associations par le BDE.\\
 
\section{Approche}
\subsection{Structure du site}
Nous avons donc pensé le site comme une plateforme épurée où la seule page accessible sans connexion est "index.php".\\
 Il y a donc 3 type d'utilisateurs : les \textbf{élèves}, les \textbf{membres du bde} et les \textbf{présidents}.\\ Chaque type d'utilisateur a accès à des fonctionnalités différentes.\\ 
 
 On a choisi d'utiliser une base de données découpée en 5 tables : 
 \begin{itemize}
 	\item users(id\_user,firstname,lastname,pseudo,year,password,mail,bde,president)
 	\item associations()
 	\item score()
 	\item pointsassos()
 \end{itemize} 

\subsection{Fonctionnalités pour les élèves}
	Les pages accessibles par tous les élèves enregistrés sont "eleves.php" qui permet aux élèves de consulter leur participations aux évènement des différentes associations et aux et "profil.php"
\section{Répartition des rôles}
\section{Problèmes rencontrés}
\end{document}